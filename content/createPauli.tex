\apisummary{
    
}

\begin{apidefinition}

\begin{Csynopsis}
    
\end{Csynopsis}

\begin{Cppsynopsis}
    
\end{Cppsynopsis}


\begin{apiarguments}
    \apiargument{IN}{sitemap}{mapping of integer 'key' to a string of value "x", "y" or "z"}
    \apiargument{IN}{paulistr}{string}
    \apiargument{IN}{coef}{coefficient}
\end{apiarguments}

\apidescription{
        \FUNC{createPauli} accepts a string representation of a complex Pauli operator, \VAR{paulistr}, or an integer:string mapping, \VAR{sitemap}, of an integer site number to a string "x", "y", or "z". The integer \VAR{coef} argument can optionally be included with \VAR{sitemap} to multiply each term in the site map. If \VAR{coef} is not included, the default value is 1.
        - if empty map, is an identity
        - map is any integer index to string x, y or z
        - coefficient optional, defaults to 1
        - string composition/concatenation of the form "<coefficient> <pauli string> <integer> ... <optional operation>"
}

\apireturnvalues{
    
}      

\apinotes{
 The interger:string mapping has a seperate datatype representation for \Clang and \Cpp. The \Clang representation is a \qcor library type, \DATATYPENAME{SiteMap}.  The \Cpp representation is the \Cpp standard library type \DATATYPENAME{map}.  
}

\begin{apiexamples}

\apicppexample
    { This is a simple program that calls \FUNC{createPauli}: } 
    { example_code/createPauli_ex.cpp} 
    {}

\end{apiexamples}

\end{apidefinition}
