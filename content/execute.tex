\apisummary{
    
}

\begin{apidefinition}

\begin{Csynopsis}
    void execute (Handle *hdl)
\end{Csynopsis}

\begin{Cppsynopsis}
    void qcor::execute (Handle *hdl)
\end{Cppsynopsis}


\begin{apiarguments}
    \apiargument{IN}{hdl}{Handle created from the initial taskInitiate call.}
\end{apiarguments}

\apidescription{
    This re-executes the quantum kernel associated after a taskInitiate and sync have already been called.  This requires a subsequent sync call to get results of a re-execution.  
    %SP 8/4/2020: I'm confused by this descption.  
    % 'the quantum kernel associated' - with what?
    % why re-execute?
    % what does it return? nothing?
}

\apireturnvalues{
    
}      

\apinotes{
    
}

\begin{apiexamples}

\apicppexample
    { This is a simple program that calls \FUNC{execute}: } 
    { example_code/execute_ex.cpp} 
    {}

\end{apiexamples}

\end{apidefinition}
