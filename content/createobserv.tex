\apisummary{
    Creates an observable
}

\begin{apidefinition}

\begin{Csynopsis}
    Observable* createObservable (PauliOperator *pauliOp, FieldOperator *fieldOp)
\end{Csynopsis}

\begin{Cppsynopsis}
    Observable* qcor::createObservable (PauliOperator &pauliOp, FieldOperator &fieldOp)
\end{Cppsynopsis}


\begin{apiarguments}
    \apiargument{IN}{pauliOp}{address of pre-defined Pauli operator}
    \apiargument{IN}{fieldOp}{address of pre-defined Field operator}
    \apiargument{OUT}{\DATATYPENAME{Observable}{observable output}
\end{apiarguments}

\apidescription{
        \FUNC{createObservable} allocates memory for a single \DATATYPENAME{Observable}. The \VAR{pauliOp} and \VAR{fieldOp} are  optional arguments that can be used to create the new Observable with a pre-defined Pauli and/or Field operator(s).
}

\apireturnvalues{
    Returns a pointer to a new \DATATYPENAME{Observable}. 
}      

\apinotes{
    
}

\begin{apiexamples}

\apicppexample
    { This is a simple program that calls \FUNC{createObservable}: } 
    { example_code/createObservable_ex.cpp} 
    {}

\end{apiexamples}

\end{apidefinition}
