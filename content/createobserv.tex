\apisummary{
    Creates an observable
}

\begin{apidefinition}

\begin{Csynopsis}
    Observable createObservable (Observable *observIn, PauliOperator *pauliOp, FieldOperator *fieldOp, AlgebraicOperation *ops)
\end{Csynopsis}

\begin{Cppsynopsis}
    Observable qcor::createObservable (Observable observIn[], PauliOperator pauliOp[], FieldOperator fieldOp[], AlgebraicOperation ops[])
\end{Cppsynopsis}


\begin{apiarguments}
    \apiargument{IN}{observIn}{array of observable}
    \apiargument{IN}{pauliOp}{array of single Pauli operators}
    \apiargument{IN}{fieldOp}{array of single Field operators}
    \apiargument{IN}{ops}{list of operations}
    \apiargument{OUT}{\DATATYPENAME{Observable}{observable output}
\end{apiarguments}

\apidescription{
        \FUNC{createObservable} creates a new \DATATYPENAME{Observable}, \VAR{observ}. The arrays \VAR{observIn}, \VAR{pauliOp}, and \VAR{fieldOp} are all optional, but a combination of one of more of these arrays must be passed to create a new \DATATYPENAME{Observable}. \VAR{ops} is also optional.  If \VAR{ops} is passed to \FUNC{createObservable}, the operations are mapped by array index to the operands listed in \VAR{observIn}, \VAR{pauliOp}, and \VAR{fieldOp}.
}

\apireturnvalues{
    Returns an \DATATYPENAME{Observable} datum. 
}      

\apinotes{
    
}

\begin{apiexamples}

\apicppexample
    { This is a simple program that calls \FUNC{createObservable}: } 
    { example_code/createObservable_ex.cpp} 
    {}

\end{apiexamples}

\end{apidefinition}
