\apisummary{
    Creates an observable
}

\begin{apidefinition}

\begin{Csynopsis}
    Observable* createObservable (SiteMap *sitemap, double complex coef)
\end{Csynopsis}

\begin{Cppsynopsis}
    Observable* qcor::createObservable (std::map<int,string> sitemap, std::complex coef)
\end{Cppsynopsis}


\begin{apiarguments}
    \apiargument{IN}{sitemap}{mapping of an integer to a string of value``x'', ``y'' or ``z''}
    \apiargument{IN}{coef}{coefficient for sitemap terms}
\end{apiarguments}

\apidescription{
        \FUNC{createObservable} accepts an integer:string mapping, \VAR{sitemap}, of an integer site number to a string ``x'', ``y'', or ``z''. \VAR{sitemap} may contain one or more integer:string mappings. The complex coefficient \VAR{coef} argument can optionally be included to multiply each term in the site map. If \VAR{coef} is not included, the default value is 1.
}

\apireturnvalues{
    Returns a pointer to a new \DATATYPENAME{Observable}. 
}      

\apinotes{
    The interger:string mapping has a seperate datatype representation for \Clang and \Cpp. The \Clang representation is a \qcor library type, \DATATYPENAME{SiteMap}.  The \Cpp representation is the \Cpp standard library type \DATATYPENAME{map}. 
 %SP 8/4/2020: space after \Clang ? 
    %EZJ: Does one need to createobserve(fieldOp) and then use observableTransform(fieldOp) to populate the observable with data?
    %i.e. we need to specify and deliniate the relationship between createobserve and observableTransform. 
}

\begin{apiexamples}

\apicppexample
    { This is a simple program that calls \FUNC{createObservable}: } 
    { example_code/createObservable_ex.cpp} 
    {}

\end{apiexamples}

\end{apidefinition}
