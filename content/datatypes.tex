The \qcor specification has a set of datatypes that enable the expression of quantum concepts within the \qcor \ac{API}s.

\medskip{}


\begin{tabular}{|l|l|l|}
\hline 
\ColHead{Datatype} & \ColHead{Description} \\
\hline 
\hline 
\DATATYPENAME{Kernel} & function pointer to a high level representation of the quantum program\\
Note: (TODO) Move the Kernel to the appropriate spec location
\hline 
\DATATYPENAME{CompositeOperator}
\hline
\end{tabular}

\medskip{}

%Enumerated Types
\DATATYPENAME{OperatorType} & 
NOTE: (TODO) enumerate different observable types 
Pauli
FieldOperator



%Composite datatype
\DATATYPENAME{ObjectiveFunction} & Captures the behavior of a parameterizable function. It is initialized with the problem-specific observable and kernel, and exposes a method to evaluate the kernel given a set of parameters and executes general pre and post processing of the kernel execution\\
\hline
\DATATYPENAME{Observable} &  operator built from the basic operators, PauliOperator and FieldOperator. Contains the data to algebraically relate basic and complex operations, and the measurements from the observation of the kernel execution.\\

\hline
\DATATYPENAME{ResultsBuffer} & contains the measurement results from a single quantum execution\\
Note: (TODO) Define the methods in the data structure.
\hline 
\DATATYPENAME{PauliOperator} & single operator type with three attributes: PauliX, PauliY or PauliZ, site/spatial index which is an integer value, coefficient float value
\hline
\DATATYPENAME{FieldOperator} & single operator type with four attributes: creation or anihilation, exchangeStatistics, site/spatial index which is an integer value, coefficient float value
\hline
\DATATYPENAME{AlgebraicOperation} & describes operations on Pauli and Field operands
\hline


