The \qcor specification has a set of datatypes that enable the expression of quantum concepts within the \qcor \ac{API}s.

\medskip{}


\begin{tabular}{|l|l|l|}
\hline 
\ColHead{Datatype} & \ColHead{Description} \\
\hline 
\hline 
\DATATYPENAME{Kernel} & function pointer to a high level representation of the quantum program\\
Note: (TODO) Move the Kernel to the appropriate spec location
\hline 
\DATATYPENAME{CompositeOperator}
\hline
\end{tabular}

\medskip{}

%Enumerated Types
\DATATYPENAME{OperatorType} & 
NOTE: (TODO) enumerate different observable types
Fermionic 
Pauli
FieldOperator
Clause/CostFunction


%Composite datatype
\DATATYPENAME{ObjectiveFunction} & Captures the behavior of a parameterizable function. It is initialized with the problem-specific observable and kernel, and exposes a method to evaluate the kernel given a set of parameters and executes general pre and post processing of the kernel execution\\
\hline
\DATATYPENAME{Observable} & stores the data to describe an observable that can be used to map an observable to another observable.  Members are an OperatorType, and a CompositeOperator.\\

\hline
\DATATYPENAME{ResultsBuffer} & contains the measurement results from a single quantum execution\\
Note: (TODO) Define the methods in the data structure.
\hline 
\DATATYPENAME{BasicPauliOperator} & type with three attributes: PauliX, PauliY or PauliZ, site/spatial index which is an integer value, coefficient float value
\DATATYPENAME{GenericOperator} & type with three attributes: PauliX, PauliY or PauliZ, site/spatial index which is an integer value, coefficient float value
