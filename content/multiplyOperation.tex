\apisummary{
     
}

\begin{apidefinition}

\begin{Csynopsis}
    Obervable* multiplyOperation(Observable* op1, Observable* op2)
    FieldOperator* multiplyOperation(FieldOperator* op1, FieldOperator* op2)
\end{Csynopsis}

\begin{Cppsynopsis}
    Obervable* qcor::multiplyOperation(Observable &op1, Observable &op2)
    FieldOperator* qcor::multiplyOperation(FieldOperator &op1, FieldOperator &op2)
\end{Cppsynopsis}


\begin{apiarguments}
    \apiargument{IN}{op1}{left operand}
    \apiargument{IN}{op2}{right operand}
\end{apiarguments}

\apidescription{
        The \FUNC{multiplyOperation} routine multiplies \VAR{op1} and \VAR{op2}.
}

\apireturnvalues{
     Returns the multiplication of \VAR{op1} and \VAR{op2}.
}      

\apinotes{
    The operand datatypes must be the same and are preserved during the multiply operation to return the same datatype as the operand.
}

\begin{apiexamples}

\apicppexample
    { This is a simple program that calls \FUNC{multiplyOperation}: } 
    { example_code/multiplyOperation_ex.cpp} 
    {}

\end{apiexamples}

\end{apidefinition}
