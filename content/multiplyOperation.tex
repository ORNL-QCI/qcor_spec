\apisummary{
     Takes Observables and retrurns their product. 
}

\begin{apidefinition}

\begin{Csynopsis}
    Obervable* multiplyOperation(Observable* op1, Observable* op2, ...)
\end{Csynopsis}

\begin{Cppsynopsis}
    Obervable* qcor::multiplyOperation(Observable &op1, Observable &op2, ...)
\end{Cppsynopsis}


\begin{apiarguments}
    \apiargument{IN}{op1}{first Observable operand}
    \apiargument{IN}{op2}{second Observable operand}
    \apiargument{IN}{...}{additional Observable operands}
\end{apiarguments}

\apidescription{
        The \FUNC{multiplyOperation} routine multiplies two or more \DATATYPENAME{Observables}.
}

\apireturnvalues{
     Returns new instance of \DATATYPENAME{Observable}.
}      

\apinotes{
    
}

\begin{apiexamples}

\apicppexample
    { This is a simple program that calls \FUNC{multiplyOperation}: } 
    { example_code/multiplyOperation_ex.cpp} 
    {}

\end{apiexamples}

\end{apidefinition}
