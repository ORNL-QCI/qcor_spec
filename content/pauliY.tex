\apisummary{
    Creates Pauli Y operator at site n.
}

\begin{apidefinition}

\begin{Csynopsis}
    Observable pauliY(int n)
\end{Csynopsis}

\begin{Cppsynopsis}
    Observable qcor::pauliY(int n)
\end{Cppsynopsis}


\begin{apiarguments}
    \apiargument{IN}{n}{integer value}
\end{apiarguments}

\apidescription{
         \FUNC{pauliY} accepts an integer input that creates a Pauli Y operator at site \VAR{n}.  The Pauli Y operator is returned as an \DATATYPENAME{Observable} type.
}

\apireturnvalues{
    Returns a single instance of \DATATYPENAME{Observable}.
}      

\apinotes{
    
}

\begin{apiexamples}

\apicppexample
    { This is a simple program that calls \FUNC{pauliY}: } 
    { example_code/pauliY_ex.cpp} 
    {}

\end{apiexamples}

\end{apidefinition}
