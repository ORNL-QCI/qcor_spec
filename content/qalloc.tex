\apisummary{
    Allocates memory for qreg.
}

\begin{apidefinition}

\begin{Csynopsis}
    qreg* qalloc(int size)
\end{Csynopsis}

\begin{Cppsynopsis}
    qreg* qcor::qalloc(int size)
\end{Cppsynopsis}


\begin{apiarguments}
    \apiargument{IN}{size}{number of qubits}
\end{apiarguments}

\apidescription{
        \FUNC{qalloc} allocates memory for qubits on the host, quantum controller and quantum device.  This routines allocates (N + C)*E bits of memory on the host, N*E bits of memory on the controller and N qubits on the quantum device, where N is the number of qubits, C is the bit size of an integer on a classical system, and E is the number of experiments to be performed on the quantum device. The number of experiments can be set at compile time or at runtime using \FUNC{set_num_experiments}.
}

\apireturnvalues{
    returns a host device pointer of type \DATATYPENAME{qreg}.
}      

\apinotes{
    
}

\begin{apiexamples}

\apicppexample
    { This is a simple program that calls \FUNC{qalloc}: } 
    { example_code/qalloc_ex.cpp} 
    {}

\end{apiexamples}

\end{apidefinition}
