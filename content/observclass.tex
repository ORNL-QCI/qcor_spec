\typedescription{
\DATATYPENAME{Observable}
}


%% fill in the rows with the appropriate member name and description
%%continue to use the \VAR command to format member names
\begin{center}
\begin{tabular}{ |p{0.43\textwidth}|p{0.43\textwidth}| } 
 \hline
 \multicolumn{2}{|c|}{\ColHead{Datatype Members}} \\
 \hline
 \hline
  \multicolumn{1}{|c|}{\ColHead{Member Name}} &  \multicolumn{1}{|c|}{\ColHead{Description}} \\ 
 \hline
  Ex \VAR{member name} & fill description \\
 \hline
\end{tabular}
\end{center}


%% When filling in the rows for this table replace make sure that the \multirow number matches the number of arguments
%% for the corresponding method, and replace the example text with the appropriate text for this data type
%%conitinue to use appropriate commands to maintain consistent formatting of names and types
\begin{center}
\begin{tabular}{ |p{0.2\textwidth}|p{0.2\textwidth}|p{0.2\textwidth}|p{0.2\textwidth}| } 
 \hline
 \multicolumn{4}{|c|}{\ColHead{Datatype Methods}} \\
 \hline
 \hline
  \multicolumn{1}{|c|}{\ColHead{Method Name}} & \multicolumn{1}{|c|}{\ColHead{Description}} & \multicolumn{1}{|c|}{\ColHead{Argument Type}} & \multicolumn{1}{|c|}{\ColHead{Argument Name}} \\ 
 \hline
 \multirow{2}{0.2\textwidth}{Ex \METHODNAME{Method 1 name}} & \multirow{2}{0.2\textwidth}{Ex. Method 1 description} & Ex.  \VARTYPE{Method 1 argument 1 type} & Ex. Method 1 \VAR{argument 1 name} \\
    & & Ex. \VARTYPE{Method 1 argument 2 type} & Ex.\VAR{Method 1 argument 2 name} \\
 \hline
\multirow{3}{0.2\textwidth}{Ex \METHODNAME{Method 2 name}} & \multirow{3}{0.2\textwidth}{Ex. Method 2 description} & Ex. \VARTYPE{Method 2 argument 1 type} & Ex. \VAR{Method 2 argument 1 name} \\
    & & Ex. \VARTYPE{Method 2 argument 2 type} & Ex. \VAR{Method 2 argument 2 name} \\
    & & Ex. \VARTYPE{Method 2 argument 3 type} & Ex. \VAR{Method 2 argument 3 name} \\
 \hline
\end{tabular}
\end{center} 


\apinotes{
    
}

