\apisummary{
    creates a Boson or Fermion feild operator
}

\begin{apidefinition}

\begin{Csynopsis}
    FieldOperator* createFieldOp(char* opType, char* relType, int siteIdx, double complex coef)
\end{Csynopsis}

\begin{Cppsynopsis}
    FieldOperator* qcor::createFieldOp(string opType, string relType, int siteIdx, std::complex coef)
\end{Cppsynopsis}


\begin{apiarguments}
    \apiargument{IN}{opType}{a string that designates if the field operator is a creation or annihilation operator}
    \apiargument{IN}{relType}{a string that designates if the commutation relations; the relations are either Fermionic or Bosonic}
    \apiargument{IN}{siteIdx}{integer site index}
    \apiargument{IN}{coef}{complex coefficient value}
\end{apiarguments}

\apidescription{
        \FUNC{createFieldOp} creates a creation or annihilation field operator that is either Fermionic or Bosonic. The creation or annihilation atrribute is set by assigning the string value ``create'' or ``annihilate'', respectively, to \VAR{opType}.  The Fermionic or Bosonic attribute is set by assigning the string value ``Fermion'' or ``Boson'', respectively, to \VAR{relType}. \FUNC{createFieldOp} must also assign an integer site index via \VAR{sizeIdx}. The complex coefficient \VAR{coef} is optional, and is 1 by default.

\apireturnvalues{
    Returns a pointer to a single instance of \DATATYPENAME{FieldOperator}.
}      

\apinotes{
    
}

\begin{apiexamples}

\apicppexample
    { This is a simple program that calls \FUNC{createFieldOp}: } 
    { example_code/createFieldOp_ex.cpp} 
    {}

\end{apiexamples}

\end{apidefinition}
