\apisummary{
    Abstraction which models general transformations on operators. 
}

\begin{apidefinition}

\begin{Csynopsis}
Observable observableTransform(FieldOperator* fieldOp, void* modifier)
\end{Csynopsis}

\begin{Cppsynopsis}
Observable observableTransform(map<FieldOperator, FieldOperator> fieldOp, void* modifier)
%SP 8/4/2020: line wrap issue in pdf
\end{Cppsynopsis}


\begin{apiarguments}
    \apiargument{IN}{fieldOp}{list of variables of FieldOperator type}
    \apiargument{IN}{modifier}{a transformation mapping generalized operators to observables. This is a function pointer that implements transformations to observable produced by \FUNC{observableTransform}. The function header should accept the Observable type}
    %SP 8/4/2020: pauli or Pauli?
    %SP 8/4/2020: so the output of observableTransform can be modified by the modifier before being returned? Is this the right way to say/do this?
    %EZJ: also confused 
    \apiargument{OUT}{oberservable}{}
\end{apiarguments}

\apidescription{
    
}

\apireturnvalues{
    use shmem tex files as a guide
}      

\apinotes{
    e.g. transform fermionic Operators to a pauli Operator.
}

\begin{apiexamples}

\apicppexample
    { This is a simple program that calls \FUNC{observableTransform}: } 
    { example_code/transform_ex.cpp} 
    {}

\end{apiexamples}

\end{apidefinition}
