\apisummary{
 Invoked to launch a quantum-classical task. 
}

\begin{apidefinition}

\begin{Csynopsis}
Handle taskInitiate(ObjectiveFunction *objective, Optimizer *optimizer, bool all_opt_data)
\end{Csynopsis}

\begin{Cppsynopsis}
Handle qcor::taskInitiate(ObjectiveFunction &objective, Optimizer &optimizer, bool all_opt_data)
\end{Cppsynopsis}


\begin{apiarguments}
    \apiargument{IN}{objective}{An objective function}
    \apiargument{IN}{optimizer}{The optimization strategy}
    \apiargument{IN}{all\_opt\_data}{A flag to indicate if intermediate data should be collected}
\end{apiarguments}

\apidescription{
The \FUNC{taskInitiat}e call is invoked on the host to launch a quantum-classical task. The \FUNC{taskInitiate} call references the objective function, \VAR{objective} and the classical optimizer, \VAR{optimizer}. After launching a task the host execution continues until the host reaches a \FUNC{sync} api call. During the execution of \FUNC{taskInitiate}, optimization iterations over \VAR{objective} parameters are performed.  Setting \VAR{all\_opt\_data} to \CONST{true} saves the parameters and objective function results at every iteration to be returned in \DATATYPENAME{ResultsBuffer} at a \FUNC{sync} api call.  Setting \VAR{all\_opt\_data} to \CONST{false} discards intermediate results and returns the optimal results.
}

\apireturnvalues{
    Returns a unique instance of \DATATYPENAME{Handle}.
}      

\apinotes{

}

\end{apidefinition}
