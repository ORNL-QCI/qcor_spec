\apisummary{
 Invoked on the host to launch a quantum kernel on the quantum device.   
}

\begin{apidefinition}

\begin{Csynopsis}
Handle taskInitiate(Observable *observable, Kernel *kernel, Objective *objective, Optimizer *optimizer)
\end{Csynopsis}

\begin{Cppsynopsis}
Handle qcore::taskInitiate(Observable *observable, Kernel *kernel, Objective *objective, Optimizer *optimizer)
\end{Cppsynopsis}


\begin{apiarguments}
    \apiargument{IN}{observable}{A rule that defines the basis for the measurement after the kernel has been executed on the quantum hardware}
    \apiargument{IN}{kernel}{The quantum kernel to be executed as part of the objective function evaluation}
    \apiargument{IN}{objective}{A parameterized user defined function transforming input data into an output result}
    \apiargument{IN}{optimizer}{Takes an objective function as input and outputs the optimal function parameters and associated function value}
\end{apiarguments}

\apidescription{
The taskInitiate call is invoked on the host to launch a quantum kernel on the quantum device. The taskInitiate call references the observable data structure, the initial quantum state, the objective function, and the classical optimizer. After a quantum kernel has been launched, host execution continues until the host reaches a sync call.    
}

\apireturnvalues{
    Handle for the quantum kernel launched by the taskInitiate call.
}      

\apinotes{
Although from a high level of abstraction, the kernel executes on the quantum
device, with variational quantum computing the underlying implementation may iterate         
execution between the quantumand host devices, with the host performing classical optimization based on 
measurements from the quantum device.   
}

\begin{apiexamples}

\apicppexample
    { This is a simple program that calls \FUNC{taskInitiate}: } 
    { example_code/taskInitiate_ex.cpp} 
    {}

\end{apiexamples}

\end{apidefinition}
