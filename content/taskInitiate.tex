\apisummary{
 Invoked on the host to launch a quantum kernel on the quantum device.   
}

\begin{apidefinition}

\begin{Csynopsis}
Handle taskInitiate(ObjectiveFunction *objective, Optimizer *optimizer)
%SP 8/4/2020: fix text wrap so it doesn't cut at odd place
\end{Csynopsis}

\begin{Cppsynopsis}
Handle qcore::taskInitiate(ObjectiveFunction &objective, Optimizer &optimizer)
\end{Cppsynopsis}


\begin{apiarguments}
    \apiargument{IN}{objective}{A parameterized user defined function transforming input data into an output result}
    %SP 8/4/2020: how is this different from objectiveFunction?  the description is basically saying it is an objective function.  This has often confused me. EZJ ditto. 
    \apiargument{IN}{optimizer}{Takes an objective function as input and outputs the optimal function parameters and associated function value}
\end{apiarguments}

\apidescription{
The taskInitiate call is invoked on the host to launch a quantum kernel on the quantum device. The taskInitiate call references the objective function, and the classical optimizer. After a quantum kernel has been launched, host execution continues until the host reaches a sync call.  
}

\apireturnvalues{
    Handle for the quantum kernel launched by the taskInitiate call.
}      

\apinotes{
The taskInitiate call kicks off a hybrid quantum-classical task. For example, 
in variational quantum computing the kernel executes on the quantum 
device and the implementation may iterate         
execution between the quantum and classical
host devices, with the host performing classical optimization based on 
measurements from the quantum device.   
}

\end{apidefinition}
