\apisummary{
    method to transform composite orperators that are pauli or field operators to a pauli observable
}

\begin{apidefinition}

\begin{Csynopsis}
Observable observableTransform(compositeOperator* compOp, void* modifier)
\end{Csynopsis}

\begin{Cppsynopsis}
Observable observableTransform(map<compositeOperator, compositeOperator> compOp, void* modifier)
\end{Cppsynopsis}


\begin{apiarguments}
    \apiargument{IN}{compOp}{list of variable of compositeOperator type}
    \apiargument{IN}{modifier}{an additional transformation on a pauli composite operator. This is a function pointer that implements additional transformations on the Pauli observable produced by \FUNC{observableTransform}. The function header should accept the Observable type}
    \apiargument{OUT}{oberservable}{}
\end{apiarguments}

\apidescription{
    
}

\apireturnvalues{
    use shmem tex files as a guide
}      

\apinotes{
    
}

\begin{apiexamples}

\apicppexample
    { This is a simple program that calls \FUNC{observableTransform}: } 
    { example_code/transform_ex.cpp} 
    {}

\end{apiexamples}

\end{apidefinition}
