\qcor targets an execution model whereby application execution is directed by the host with access to an attached quantum device. Applications are broadly thought of as containing two components: a main, classical part, and one or more quantum kernels or subroutines. Figure \ref{fig:exec_model} illustrates \qcor's execution model. Designated quantum kernels are compiled and offloaded to the quantum device. The quantum device executes \qcor library calls and/or \qcor regions identified by directive notations. When the host encounters a quantum kernel, the quantum kernel, in the form of compiled hardware native or simulator 
instructions, is passed to the quantum device controller. Execution on the host is asynchronous to execution on the quantum device. 

Currently, \qcor exposes two primary library calls to the user that enable a wide variety of hybrid quantum-classical use cases: the \FUNC{taskInitiate} and \FUNC{sync} calls. The host launches a task for execution on the quantum device through the \FUNC{taskInitiate} call and synchronizes execution, via the \FUNC{sync} call, between the host and device to update data structures relative the executed quantum kernel. 

\begin{figure*}
 \centering
 \includegraphics[width=5in,height=4in]{figures/Execution_Model_Illustration_v3.png}
  \caption{Diagram of \qcor Execution Model}
  \label{fig:exec_model}
\end{figure*}

