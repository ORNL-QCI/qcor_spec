In the \ac{HQCC} paradigm, a classically hard computational task is broken up into a sequence of quantum and classical subtasks. The former are designed such that a quantum device can efficiently evaluate them. The latter are chosen so as to be solvable by a classical computer. 
\qcor leverages the \ac{HQCC} paradigm to develop a programming model that enables a sequence of quantum and classical subtasks to be implemented cooperatively in a single application. The \qcor programming model is a single-source heterogeneous programming model that allows quantum kernels to be implemented within a \Clang or \Cpp application. \qcor's language extensions allow quantum kernels to be constructed using high level \CorCpp syntax via library calls and directives. The library calls and directives are designed to satisfy the requirements for \ac{HQCC} algorithms, but could in principle be used for more general schemes, e.g. those involving feed-forward logic. 
